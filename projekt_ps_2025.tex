
\documentclass[11pt]{article}
\def\documentlanguage{slovak}
\usepackage{framed}
\usepackage{fancyvrb}
\usepackage{color}
\usepackage{listings}
\usepackage{ifpdf}
\usepackage{ifxetex}
\usepackage{graphicx}
\usepackage{longtable}
\usepackage{hyperref}
\usepackage{tcolorbox}
\ifpdf %pdftex
\usepackage[utf8]{inputenc}
\usepackage[T1]{fontenc}
\usepackage[all,pdf,2cell]{xy}\UseAllTwocells\SilentMatrices
\usepackage[\documentlanguage]{babel}
\fi
\ifxetex %xetex
\usepackage[all,pdf,2cell]{xy}\UseAllTwocells\SilentMatrices
\usepackage{polyglossia}
\begingroup\edef\x{\endgroup\noexpand\setdefaultlanguage{\documentlanguage}}\x%
\fi
\usepackage{amsthm}
\usepackage{amsmath}
\usepackage{amsfonts}
\newtheorem{theorem}{Theorem}[section]
\newtheorem{lemma}[theorem]{Lemma}
\newtheorem{proposition}[theorem]{Proposition}
\theoremstyle{definition}
\newtheorem{definition}[theorem]{Definition}
\newtheorem{example}[theorem]{Príklad}
\newcommand{\newcategory}[1]{\expandafter\newcommand\csname #1\endcsname{\mathbf{#1}}}
\input{pygments}
\begin{document}
\title{Projekt na získanie zápočtu z predmetu Počítačové siete\\LS 2024/25}
\maketitle
\section{Získanie zápočtu}
\subsection{Formálne aspekty}

\begin{tcolorbox}[colback=red!5!white,colframe=red!75!black]
Vaše repozitáre musia byť súkromné (private), nie verejné (public). Predídeme
tým nepríjemným nedorozumeniam ohľadom autorstva.
\end{tcolorbox}

\begin{itemize}
\item Odovzdávať sa bude prostredníctvom githubu, sprístupnením repozitára
menom \texttt{reader} užívateľovi \texttt{gjenca}
\item Repozitár bude obsahovať iba jeden súbor so zdrojovým kódom, nič iné.
\item Termín odovzdania projektu je 12.4.2022 23:59:59.
\item Po tomto termíne budú projekty vyhodnotené, bude oznámený poočet
bodov.
\item V prípade, že niekto bude mať záujem o zvýšenie počtu bodov, bude
vypísaný ďalší termín dokedy budú môcť záujemcovia odovzdať druhú verziu
programu.
\end{itemize}

\subsection{Kritériá hodnotenia}
\begin{itemize}
\item Posudzovať sa bude korektnosť implementácie v zmysle špecifikácie protokolu.
\item Posudzovať sa bude aj zdrojový kód: pokiaľ niekto odovzdá príliš škaredý kód,
dostane menej bodov.
\item Používajte, (adekvátne účelu) prostriedky Pythonu: funkcie, slovníky, triedy, zabudované dátové typy.
\end{itemize}

\subsection{Testovač zadania}

Na URL \url{https://github.com/gjenca/test_reader.git} je malý
testovací systém pre testovanie vášho riešenia. 
Pokyny pre použitie viď tam. Určite ho použite predtým, ako mi pošlete riešenie.

Ak nájdete nejakú chybu v tých testoch (snažil som sa ich písať robustne, ale je možné,
že tom v nich spravil *ja* nejakú chybu), nahláste mi to cez issue tracker na
URL \url{https://github.com/gjenca/test_reader/issues}, alebo mi pošlite email.

POZOR! Testy sa budú časom meniť
a pribúdať, tak ako budem Vaše zadania opravovať.

\section{Zadanie}

Napíšte forkujúci sa TCP server, ktorý bude vykonávať službu zapisovania a čítania
riadkov z textových súborov z podadresára \texttt{data} aktuálneho adresára.

\section{Protokol}

\subsection{Základné pravidlá}

\begin{itemize}
\item Protokol bude pre texty používať kódovanie UTF-8, koniec riadku je {\tt
<LF>} (ASCII kód 10), ako v Unixe.
\item Bude postavený nad TCP.
\item Klient bude serveru posielať \emph{požiadavky} (anglicky \emph{request}) a server po každej požiadavke pošle
naspäť \emph{odpoveď} (anglicky \emph{response}).
\end{itemize}
\subsection{Požiadavka}
\begin{itemize}
\item Požiadavka bude vždy obsahovať v prvom riadku \emph{metódu} (anglicky method), t.j. jedno slovo z množiny slov 
\texttt{READ, LS, LENGTH}.
\item Nasledovať bude bližšia špecifikácia požiadavky, pomocou niekoľkých \emph{hlavičiek}
(headers). Hlavičky nemusia byť uvedené (môže ich byť 0). Každá hlavička je na jednom
riadku. Hlavička sa skladá z \emph{identifikátora} a \emph{hodnoty}, sú oddelené dvojbodkou.
Identifikátor musí byť ASCII reťazec bez bielych znakov a nesmie obsahovať znak \texttt{:}
(dvojbodka).
\footnote{Viď metódy \texttt{isascii} a \texttt{isspace} v dokumentácii Pythonu}
Príklad hlavičky:
\begin{flushleft}
\texttt{File:list\_of\_students.txt}
\end{flushleft}
\item Po hlavičkách bude nasledovať vždy práve jeden prázdny riadok, ktorý značí koniec hlavičiek.
\end{itemize}

\subsection{Odpoveď}
\begin{itemize}
\item Odpoveď bude mať v prvom riadku \emph{stav} (anglicky status).
\item Stav obsahuje ako prvé slovo trojciferné číslo, nasledované práve jednou medzerou a popisom. Príklad stavu:
\begin{flushleft}
\texttt{200 Bad request}
\end{flushleft}
\item Nasledujú hlavičky, v rovnakom tvare ako pri požiadavke.
\item Potom bude vždy práve jeden prázdny riadok.
\item Nasledovať obsah odpovede (alebo nič), dĺžka vyplýva z metódy a hlavičiek (\texttt{Lines}). Obsah sú vždy nula alebo viac riadkov.
\item Spojenie za normálnych okolností ukončí klient, zavretím socketu.
\end{itemize}

\subsection{Metódy}
Teraz ideme špecifikovať po jednotlivých metódach typy rôznych požiadaviek, možných odpovedí a tomu
zodpovedajúcich efektov a prípadne obsahov odpovedí.
Ak nie je v tabuľke v poslednom stĺpci uvedený žiaden obsah odpovede, obsah odpovede
je vtedy prázdny.
\subsubsection{ READ }
\noindent\textbf{Hlavičky:}
\begin{itemize}
\item
\begin{description}
\item[Názov hlavičky]\texttt{ File }
\item[Popis hlavičky] Názov textového súboru odkiaľ treba prečítať obsah. Nesmie obsahovať znak \texttt{/}. Súbor sa má nachádzať v podadresári \texttt{data}. 
\end{description}
\item
\begin{description}
\item[Názov hlavičky]\texttt{ From }
\item[Popis hlavičky] Nezáporné prirodzené číslo. Prvý riadok, ktorý chceme prečítať, číslovanie riadkov začína od 0. 
\end{description}
\item
\begin{description}
\item[Názov hlavičky]\texttt{ To }
\item[Popis hlavičky] Nezáporné prirodzené číslo. Riadok po poslednom riadku, ktorý chceme prečítať. \texttt{From} a \texttt{To} majú teda rovnakú sémantiku ako funkcia \texttt{range} v Pythone, alebo vystrihovanie zo zoznamov v Pythone. 
\end{description}
\end{itemize}
\noindent\textbf{Správanie sa v rôznych situáciách:}
\begin{itemize}
\item
\begin{description}
\item[Situácia]
V hlavičkách sú neprípustné hodnoty, alebo niektorá hlavička chýba. Toto zahŕňa aj situáciu, keď je hodnota \texttt{From} väčšia ako hodnota \texttt{To}.
\item[Odpoveď (status)]
\texttt{ 200 Bad request }
\item[Odpoveď (hlavičky)]
Žiadne hlavičky odpovede.
\item[Odpoveď (obsah)]
Prázdny obsah odpovede.
\end{description}
\item
\begin{description}
\item[Situácia]
Požaduje sa viac riadkov, ako je v súbore.
\item[Odpoveď (status)]
\texttt{ 201 Bad line number. }
\item[Odpoveď (hlavičky)]
Žiadne hlavičky odpovede.
\item[Odpoveď (obsah)]
Prázdny obsah odpovede.
\end{description}
\item
\begin{description}
\item[Situácia]
Súbor neexistuje (\texttt{FileNotFoundError}).
\item[Odpoveď (status)]
\texttt{ 202 No such file }
\item[Odpoveď (hlavičky)]
Žiadne hlavičky odpovede.
\item[Odpoveď (obsah)]
Prázdny obsah odpovede.
\end{description}
\item
\begin{description}
\item[Situácia]
Súbor sa nedá prečítať z iných dôvodov (\texttt{OSError}).
\item[Odpoveď (status)]
\texttt{ 203 Read error }
\item[Odpoveď (hlavičky)]
Žiadne hlavičky odpovede.
\item[Odpoveď (obsah)]
Prázdny obsah odpovede.
\end{description}
\item
\begin{description}
\item[Situácia]
Normálny stav
\item[Odpoveď (status)]
\texttt{ 100 OK }
\item[Odpoveď (hlavičky)]
\texttt{Lines:} \emph{dĺžka odpovede v riadkoch}
\item[Odpoveď (obsah)]
Riadky zo súboru.
\end{description}
\end{itemize}

\subsubsection{Pomôcky}
Pre otváranie súboru pre čítanie použite \texttt{with/open} idióm,
aby ste nemuseli myslieť na to, že ho treba zavrieť.
\begin{framed}
\begin{Verbatim}[commandchars=\\\{\}]
\PY{o}{\PYZgt{}\PYZgt{}}\PY{o}{\PYZgt{}} \PY{k}{with} \PY{n+nb}{open}\PY{p}{(}\PY{l+s+s1}{\PYZsq{}}\PY{l+s+s1}{data/a.txt}\PY{l+s+s1}{\PYZsq{}}\PY{p}{)} \PY{k}{as} \PY{n}{f}\PY{p}{:}
\PY{o}{.}\PY{o}{.}\PY{o}{.}     \PY{n}{lines}\PY{o}{=}\PY{n}{f}\PY{o}{.}\PY{n}{readlines}\PY{p}{(}\PY{p}{)}
\end{Verbatim}

\end{framed}
\subsubsection{ LS }
\noindent\textbf{Hlavičky:}
Žiadne

\noindent\textbf{Správanie sa v rôznych situáciách:}
\begin{itemize}
\item
\begin{description}
\item[Situácia]
Normálny stav
\item[Odpoveď (status)]
\texttt{ 100 OK }
\item[Odpoveď (hlavičky)]
\texttt{Lines: $n$}, kde $n$ je počet správ/súborov (riadkov odpovede).
\item[Odpoveď (obsah)]
Názvy všetkých správ/súborov adresári \texttt{data}.
\end{description}
\end{itemize}

\subsubsection{Pomôcky}
Zoznam súborov adresári zistíte pomocou \texttt{os.listdir}.
\begin{framed}
\begin{Verbatim}[commandchars=\\\{\}]
\PY{o}{\PYZgt{}\PYZgt{}}\PY{o}{\PYZgt{}} \PY{k+kn}{import} \PY{n+nn}{os}
\PY{o}{\PYZgt{}\PYZgt{}}\PY{o}{\PYZgt{}} \PY{n}{os}\PY{o}{.}\PY{n}{listdir}\PY{p}{(}\PY{l+s+s1}{\PYZsq{}}\PY{l+s+s1}{data}\PY{l+s+s1}{\PYZsq{}}\PY{p}{)}
\PY{p}{[}\PY{l+s+s1}{\PYZsq{}}\PY{l+s+s1}{b.txt}\PY{l+s+s1}{\PYZsq{}}\PY{p}{,} \PY{l+s+s1}{\PYZsq{}}\PY{l+s+s1}{a.txt}\PY{l+s+s1}{\PYZsq{}}\PY{p}{]}
\end{Verbatim}

\end{framed}
\subsubsection{ LENGTH }
\noindent\textbf{Hlavičky:}
\begin{itemize}
\item
\begin{description}
\item[Názov hlavičky]\texttt{ File }
\item[Popis hlavičky] Názov textového súboru ktorého počet riadkov chceme zistiť. Nesmie obsahovať znak \texttt{/}. Súbor sa má nachádzať v podadresári \texttt{data}. 
\end{description}
\end{itemize}
\noindent\textbf{Správanie sa v rôznych situáciách:}
\begin{itemize}
\item
\begin{description}
\item[Situácia]
V hlavičkách sú neprípustné hodnoty, alebo niektorá hlavička chýba.
\item[Odpoveď (status)]
\texttt{ 200 Bad request }
\item[Odpoveď (hlavičky)]
Žiadne hlavičky odpovede.
\item[Odpoveď (obsah)]
Prázdny obsah odpovede.
\end{description}
\item
\begin{description}
\item[Situácia]
Súbor neexistuje (\texttt{FileNotFoundError}).
\item[Odpoveď (status)]
\texttt{ 202 No such file }
\item[Odpoveď (hlavičky)]
Žiadne hlavičky odpovede.
\item[Odpoveď (obsah)]
Prázdny obsah odpovede.
\end{description}
\item
\begin{description}
\item[Situácia]
Súbor sa nedá prečítať z iných dôvodov (\texttt{OSError}).
\item[Odpoveď (status)]
\texttt{ 203 Read error }
\item[Odpoveď (hlavičky)]
Žiadne hlavičky odpovede.
\item[Odpoveď (obsah)]
Prázdny obsah odpovede.
\end{description}
\item
\begin{description}
\item[Situácia]
Normálny stav
\item[Odpoveď (status)]
\texttt{ 100 OK }
\item[Odpoveď (hlavičky)]
\texttt{Lines:} 1
\item[Odpoveď (obsah)]
Jeden riadok, obsahujúci počet riadkov v tom súbore.
\end{description}
\end{itemize}

\subsubsection{Pomôcky}
\subsubsection{Neznáma metóda}
Ak klient pošle na začiatku požiadavky čokoľvek iné, ako je
riadok obsahujúci známu metódu, server vráti status \texttt{204 Unknown method} s
prázdnymi hlavičkami a obsahom a {\em uzavrie spojenie}.
\subsection{Príklad komunikácie klient-server}
\texttt{C->S} sú označené dáta zasielané klientom serveru, \texttt{S->C} naopak. Všimnite si aj prázdne riadky označujúce koniec hlavičiek, sú súčasťou správnej implementácie protokolu.
\begin{framed}
\begin{Verbatim}[commandchars=\\\{\}]
C\PYZhy{}\PYZgt{}S LS
C\PYZhy{}\PYZgt{}S 
S\PYZhy{}\PYZgt{}C 100 OK
S\PYZhy{}\PYZgt{}C Lines:2
S\PYZhy{}\PYZgt{}C
S\PYZhy{}\PYZgt{}C b.txt
S\PYZhy{}\PYZgt{}C a.txt
C\PYZhy{}\PYZgt{}S LENGTH
C\PYZhy{}\PYZgt{}S File:a.txt
C\PYZhy{}\PYZgt{}S 
S\PYZhy{}\PYZgt{}C 100 OK
S\PYZhy{}\PYZgt{}C Lines:1
S\PYZhy{}\PYZgt{}C 
S\PYZhy{}\PYZgt{}C 3
C\PYZhy{}\PYZgt{}S READ
C\PYZhy{}\PYZgt{}S File:a.txt
C\PYZhy{}\PYZgt{}S From:0
C\PYZhy{}\PYZgt{}S To:2
C\PYZhy{}\PYZgt{}S 
S\PYZhy{}\PYZgt{}C 100 OK
S\PYZhy{}\PYZgt{}C Lines:2
S\PYZhy{}\PYZgt{}C 
S\PYZhy{}\PYZgt{}C This is line 1 of a.txt
S\PYZhy{}\PYZgt{}C This is line 2 of a.txt
C closes connection
\end{Verbatim}

\end{framed}
\end{document}
