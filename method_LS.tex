\subsubsection{ LS }
\noindent\textbf{Hlavičky:}
Žiadne

\noindent\textbf{Správanie sa v rôznych situáciách:}
\begin{itemize}
\item
\begin{description}
\item[Situácia]
Normálny stav
\item[Odpoveď (status)]
\texttt{ 100 OK }
\item[Odpoveď (hlavičky)]
\texttt{Lines: $n$}, kde $n$ je počet správ/súborov (riadkov odpovede).
\item[Odpoveď (obsah)]
Názvy všetkých správ/súborov adresári \texttt{data}.
\end{description}
\end{itemize}
