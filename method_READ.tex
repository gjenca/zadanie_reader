\subsubsection{ READ }
\noindent\textbf{Hlavičky:}
\begin{itemize}
\item
\begin{description}
\item[Názov hlavičky]\texttt{ File }
\item[Popis hlavičky] Názov textového súboru odkiaľ treba prečítať obsah. Nesmie obsahovať znak \texttt{/}. Súbor sa má nachádzať v podadresári \texttt{data}. 
\end{description}
\item
\begin{description}
\item[Názov hlavičky]\texttt{ From }
\item[Popis hlavičky] Nezáporné prirodzené číslo. Prvý riadok, ktorý chceme prečítať, číslovanie riadkov začína od 0. 
\end{description}
\item
\begin{description}
\item[Názov hlavičky]\texttt{ To }
\item[Popis hlavičky] Nezáporné prirodzené číslo. Riadok po poslednom riadku, ktorý chceme prečítať. \texttt{From} a \texttt{To} majú teda rovnakú sémantiku ako funkcia \texttt{range} v Pythone, alebo vystrihovanie zo zoznamov v Pythone. 
\end{description}
\end{itemize}
\noindent\textbf{Správanie sa v rôznych situáciách:}
\begin{itemize}
\item
\begin{description}
\item[Situácia]
V hlavičkách sú neprípustné hodnoty, alebo niektorá hlavička chýba. Toto zahŕňa aj situáciu, keď je hodnota \texttt{From} väčšia ako hodnota \texttt{To}.
\item[Odpoveď (status)]
\texttt{ 200 Bad request }
\item[Odpoveď (hlavičky)]
Žiadne hlavičky odpovede.
\item[Odpoveď (obsah)]
Prázdny obsah odpovede.
\end{description}
\item
\begin{description}
\item[Situácia]
Požaduje sa viac riadkov, ako je v súbore.
\item[Odpoveď (status)]
\texttt{ 201 Bad line number. }
\item[Odpoveď (hlavičky)]
Žiadne hlavičky odpovede.
\item[Odpoveď (obsah)]
Prázdny obsah odpovede.
\end{description}
\item
\begin{description}
\item[Situácia]
Súbor neexistuje (\texttt{FileNotFoundError}).
\item[Odpoveď (status)]
\texttt{ 202 No such file }
\item[Odpoveď (hlavičky)]
Žiadne hlavičky odpovede.
\item[Odpoveď (obsah)]
Prázdny obsah odpovede.
\end{description}
\item
\begin{description}
\item[Situácia]
Súbor sa nedá prečítať z iných dôvodov (\texttt{OSError}).
\item[Odpoveď (status)]
\texttt{ 203 Read error }
\item[Odpoveď (hlavičky)]
Žiadne hlavičky odpovede.
\item[Odpoveď (obsah)]
Prázdny obsah odpovede.
\end{description}
\item
\begin{description}
\item[Situácia]
Normálny stav
\item[Odpoveď (status)]
\texttt{ 100 OK }
\item[Odpoveď (hlavičky)]
\texttt{Lines:} \emph{dĺžka odpovede v riadkoch}
\item[Odpoveď (obsah)]
Riadky zo súboru.
\end{description}
\end{itemize}
